% 10/06/2017 
%LaTex version of my CV

\documentclass[10pt,a4paper]{article}
\usepackage[left=0.7in, top=0.6in, right=0.7in, bottom=0.6in]{geometry}
\usepackage{enumerate} % put in numbers or bullet points
\usepackage{setspace}	% line spacing					
\usepackage{eurosym}
%\usepackage{fullpage}
\usepackage{fancyhdr}
\pagestyle{fancy} % page numbers and headers and footers


\fancyhead[RO,RE]{Dr Kevin Healy CV}
\fancyfoot[C]{\thepage~of 7}

\renewcommand{\headrulewidth}{0.1pt}
\renewcommand{\footrulewidth}{0pt}

\renewcommand*{\familydefault}{\sfdefault} %% I changed the font
\usepackage{helvet}

%\usepackage[osf]{mathpazo} % palatino font package
%\usepackage{fontspec} %this must be in XeLaTeX
%\setmainfont{Verdana}%
%\setmainfont{Calibri}%
%\setsansfont{Tahoma}%
%\setmainfont{Helvetica}%
%\setmainfont{Garamond}%
\usepackage{hyperref} % allows the inclusion of hyperlinks




\usepackage{titlesec} % Used to customize the \section command
\titleformat{\section}{\Large\raggedright}{}{0em}{} [\titlerule] % Text formatting of sections
\titlespacing{\section}{0pt}{3pt}{10pt} % Spacing around sections (left, above, below)

\begin{document}

\par{\centering{\Huge Dr Kevin {Healy}}\smallskip\par}

\large\centering{{Marie Curie Research Fellow at St Andrews University}}\\
\smallskip

\par{\normalsize\centering{{I am a Marie Curie research fellow interested in the comparative biology, ecology and evolution of animals. My main research centers around life history evolution and trophic ecology.}}\bigskip\par}


\begin{minipage}[t]{0.5\textwidth}
\raggedright

Email:  healyke@tcd.ie\\
Tel: \hspace{0mm}+353 851557282\\
Address: School of Biology,\\
\hspace{0mm}Sir Harold Mitchell Building,\\
\hspace{0mm}Greenside Place, St Andrews,.\\ 
\hspace{0mm}St Andrews, UK.\\ 

\end{minipage}
\begin{minipage}[t]{0.45\textwidth}

Website: \href{http://healyke.github.io}{healyke.github.io}\\
Google Scholar: \href{http://scholar.google.com/citations?user=5Kb9u8EAAAAJ}{link}\\
ORCID: \href{http://orcid.org/0000-0002-3548-6253}{0000-0002-3548-6253}\\
ResearcherID: \href{http://www.researcherid.com/rid/H-6512-2013}{H-6512-2013}\\
Twitter: \href{https://twitter.com/healyke}{@healyke}\\
\end{minipage}

\bigskip

\section{Education and Academic Career}

\raggedright	
\textbf{2017 - 2019:} Marie Curie Research Fellow, The University of St Andrews, Scotland.
 \smallskip
\par{\fontsize{10.5}{10} I am current working on the SCAVENGER project which focuses on using agent based models and comparative methods to understand the drivers of scavenging behaviors in a range of animals and environmental conditions.
%Supervised by Prof. Graeme Ruxton (The University of St Andrews).
\bigskip}

\raggedright	
\textbf{2015 - 2017:} Research Fellow in Zoology, Trinity College Dublin.
 \smallskip
\par{\fontsize{10.5}{10} This position focused on using the animal demography dataset COMADRE to test questions regarding how body size, ancestral relationships and ecology determines species life history strategies. Supervised by Prof. Yvonne Buckley (Trinity College Dublin).\bigskip}

\raggedright	
\textbf{2011 - 2015:} Ph.D. in Zoology, Trinity College Dublin. Title: “Predator-prey allometry across body size and interaction dimensionality”.\\ \par{\fontsize{10.5}{10} I investigated how various ecological and physiological traits, including visual perception, species lifespan and predatory traits, such as venom, define predator-prey interactions using comparative approaches across vertebrates. Supervised by Dr. Andrew Jackson and Dr. Andrew Parnell.
\bigskip}

%------------------------------------------

\textbf{2007-2011:} B.A. Mod in Zoology, First class honours and Gold Medal winner, Trinity College Dublin.
Thesis: "Fractal structure of intestinal parasite communities in field mice". (Overall mark of 82\%).
I investigated the body size distribution of intestinal parasites in the wood mouse (\textit{Apodemus sylvaticus}). This involved trapping in the field, dissection work, parasite identification, preservation and data analysis.
\bigskip

%------------------------------------------
\textbf{2010:} Ureka research position in SoMER program, National College of Ireland Maynooth.\\
\par{\fontsize{10.5}{10} Ten week program under the supervision of Dr. Christen Griffen investigating the evolutionary divergence of entomopathogenic nematodes. This work involved the construction of a phylogeny for a groups of entomopathogenic nematodes collected from Bull island Dublin using molecular techniques.}
\bigskip

\textbf{2006 and 2007:} General archaeological operator for Valerie J. Keeley Ltd.\\
\par{\fontsize{10.5}{10} I worked as a general operator on site for three months in 2006 and worked both on site
and in post excavation for three months in 2007.
}
\bigskip


%	Awards and Grants
%----------------------------------------------------------------------------------------

\section{Awards and Grants}


\begin{tabular}{ll}
\textbf{2017:} & Marie Skłodowska-Curie Individual Fellowship.\\
& Project Title SCAVENGER. Funded by Horizon 2020 (\euro 183,454.80)\\
\textbf{2017:} & Government of Ireland Postdoctoral Fellowship.\\
& Funded by the Irish Research Council (\euro 91,330)\\
& I declined this award in order to accept the Marie Curie Individual Fellowship.\\
\textbf{2014:} & Gordon Research Seminar mentoring program position. Funded by Gordon\\
& Research Conferences and the National Science Foundation. (\euro 1,300)\\

\textbf{2016:} & Winner of the postdoc catagory of the School of Natural\\
& Sciences lightning talks.\\
\textbf{2014:} & Awarded runner up in both the School of Natural Sciences postgraduate lightning\\
& talks and at the Zoology and Botany postgraduate symposium.\\
\textbf{2014:} & Awarded runner up in the "I'm a scientist get me out of here" outreach compitition\\
\textbf{2014:} & Named top contributer to TCD ecology and evolution discussion group "NERD club".\\
\textbf{2011:} & Awarded Gold medal by TCD for “exceptional merit at degree examinations”\\
& in final year of B.A Mod. Zoology by coming first in class and achieving an\\
& overall final year mark of 77\%.\\
\end{tabular}

\bigskip


\clearpage
%-------------------------------------
% Skills
%--------------------------------------
\bigskip
\section{Skills}

\raggedright\textbf{Field, Communication and Laboratory skills}\\
\begin{tabular}{ll}
%---------------------------------------
\textbullet & Use of museum materials for public events such as part of TCD Research Night.\\
\textbullet & Species identification skills in helminth parasites.\\
\textbullet & Experience in small mammal trapping\\
\textbullet & Fieldwork experience in archaeological excavation.\\
\textbullet & Image and video production using graphics software including Inkscape, GIMP and ImageJ.\\
\textbullet & Document typesetting using LaTex.\\
\textbullet & Data management using the version control software GitHub\\
\textbullet & Public speaking; including on radio, television and public events such as TEDxUCD.\\ 
\textbullet & Molecular techniques including DNA extraction PCR gained during UREKA program.\\
&\\
\end{tabular}

\raggedright\textbf{Quantitative skills}\\

\begin{tabular}{ll}
%---------------------------------------
\textbullet & Modelling and Statistical analysis in R, in particular phylogenetic comparative methods\\
\textbullet & Bayesian modelling using JAGS software and High performance computing using\\
&UNIX based parallel computing in the Trinity Centre for High performance Clusters.\\
\textbullet & Individual based modelling using Netlogo software.\\

&\\
\end{tabular}

\bigskip

%----------------------------------------------------------------------------------------
%	Academic outreach
%----------------------------------------------------------------------------------------

\section{Outreach, teaching and academic service}
\raggedright\textbf{Outreach}\\
\begin{tabular}{ll}
%---------------------------------------
\textbullet& I have co-organised three \href{http://discoverresearchdublin.com/2014/08/20/night-life/}{Discover Research Night} events in the TCD Zoology Museum\\ 
&aimed at communicating research in evolution and ecology to the general public.\\
&These events have attracted a combined attendance of over 600.\\
\textbullet& Public speaking including a TEDxUCD talk on lifespan evolution, talks in Science\\
& Gallery Dublin on subjects including venom evolution and bioluminescence.\\
\textbullet& I have also contributed science communication pieces for the Journal of Animal\\ 
& Ecology \href{https://journalofanimalecology.wordpress.com/2017/09/23/high-society-the-social-network-of-vultures/}{(link)} and to the \href{http://www.ecoevoblog.com/}{ EcoEvo blog}, with one of my posts reaching the semifinal\\
& stages of the \href{http://www.3quarksdaily.com/3quarksdaily/2014/09/3qd-science-prize-semifinalists-2014.html}{3 quirks daily science blog awards}.\\
\textbullet & I also produce short videos and images as tools to communicate my research \href{http://healyke.github.io/outreach.html}{(see website)}.\\
\textbullet & I have been involved in numerous outreach events including BioBlitz events, PubPhD,\\
&Soapbox Science and I was a finalist in the 2014 "I'm a scientist get me out of here" event.\\
\textbullet & Postgraduate Representative for the Zoology Department 2014-15.\\
\end{tabular}

\raggedright\textbf{Professional society affiliation}\\
\begin{tabular}{ll}
%---------------------------------------
\textbullet& I am currently a committe member of the Irish Ecological Association (IEA).\\
& and the British Ecological Society (BES) Macroecology special interest group.\\
&I am also a member of the European Society for Evolutionary Biology (ESEB), and the\\ 
&American Society of naturalists\\
\end{tabular}

\raggedright\textbf{Reviewing}\\
\begin{tabular}{ll}
%---------------------------------------
\textbullet&I regularly act as a reviewer for major international academic journals including Ecology\\
&Letters, Current Biology, Journal of Animal Ecology, Perspectives in Plant Ecology,\\
&Scientific Reports, Evolution and Systematics, Animal Behaviour,\\
&Rapid Communications in Mass Spectrometry,the Journal of Biogeography,\\
&and Proceedings of the Royal Society B.\\
&See my \href{https://publons.com/a/1187955/}{publons} reviewer profile for more information.\\

\textbullet&I have also acted as a reviewer for research grants awarded by The German Israeli Foundation\\
&for Scientific Research and Development.


\end{tabular}

\bigskip

\raggedright\textbf{Teaching Experience}\\
\begin{tabular}{ll}
%---------------------------------------
\textit{Teaching and Tutorials}:&I have taught courses and workshops for Undergraduate, Masters\\
& and Postgraduate level students. Classes include;\\
& Macroevolution as part of the 4th year Zoology evolution Module;\\
& Introduction to Evolution for 2nd year science students;\\
& Research comprehension for final year Zoology students.\\
& Statistics for the TCD masters course in Biodiversity and Conservation;\\
& Comparative analysis for postgraduate students in University College\\
& Cork, Trinity College Dublin and the Max Plank institute in Rostock, Germany.\\
\end{tabular}


\begin{tabular}{ll}
\textit{Field Course Assistant}:& I was a field assistant for a week long ecology field course\\ 
&for 3rd year zoology students teaching field skills in small mammal\\
&trapping, insect and bird identification and other general ecology\\
&field skills.
\end{tabular}

\begin{tabular}{ll}
\textit{Project supervision}:&\hspace{5.7mm} I have supervised several final year zoology student thesis projects\\

\end{tabular}



%-------------------------------------
% Conferences and workshops
%--------------------------------------


\section{Conferences and invited presentations}

\raggedright
\begin{tabular}{ll}
%-------------------------------
\textbf{11/08/2017:} & Attended the annual ESA conference in Portland, Oregon, USA. I presented my talk\\ 
& "Mapping animal life-history strategies using the COMADRE database".\\ 
%-------------------------------
\textbf{7/12/2016:} & Invited seminar speaker to the Animal and Plant Sciences Department in The\\
& University of Sheffield. I presented my talk "Mapping animal life-history\\
& strategies using the COMADRE database".\\
%-------------------------------
\textbf{14/12/2016:} & Attended the British Ecological Society in Liverpool, UK. I presented my talk\\ 
& "Mapping animal life-history strategies using the COMADRE database".\\ 
%-------------------------------
\textbf{3/10/2016:} & Attended the EvoDemos conference at the University of Virgina US. I presented\\ 
& my talk "Mapping animal life-history strategies using the COMADRE database".\\
%-------------------------------
\textbf{4/11/2015:} & TEDxUCD invited speaker were I presented my talk\\
& \href{https://www.youtube.com/watch?v=-CHtfWEKifY}{"Listening to evolutionary oddities"}.\\ 
%-------------------------------
\textbf{9/10/2015:} & Invited speaker to the Dublin Science Gallery Cáfe Dark Secrets event.\\ 
& I presented my talk "BIOLUMINESCE: How living organisms produce and\\ 
& emit light".\\
%-------------------------------
\textbf{16/12/2015:} & Attended the British Ecological Society Annual meeting in Edinburgh. Were\\ 
& I presented my talk "Venom evolution in snakes; body size, habitat\\
& dimensionality and a diet of eggs".\\
%-------------------------------
\textbf{19/7/2014:} & Attended the Gordon Research Seminar "Unifying Ecology Across Scales".\\ 
& were I gave a talk on my theropod research entitled "A tail of two extremes".\\
%-------------------------------
\textbf{2014:} & I was the keynote student speaker at the BES Macroecology meeting Nottingham.\\ 
& with my talk "Ecology and mode-of-life explain lifespan variation in birds and mammals".\\
%-------------------------------
\textbf{2014:} & I was an invited speaker for the Irish Longitudinal Study on Aging group (TILDA).\\ 
& With my talk "Ecology and mode-of-life explain lifespan variation in birds and mammals".\\
%-------------------------------
\textbf{2014:} & I was an invited speaker to the Dublin Science Gallery Cafe DEAD BEATS event.\\ 
& with my talk on my snake venom evolution research "Why so venomous?".\\
%-------------------------------
\textbf{2013:} & Attended the ESEB XIV Congress, Lisbon, Portugal. Were I presented\\
& my talk "Metabolic rate and body size linked with perception of temporal information"\\
%-----------------------------
\textbf{2013:} & Attended the British Ecological Society Macroecology SIG meeting in Sheffield.\\
& I presented my talk "Metabolic rate and body size linked with perception of temporal information".\\
%-------------------------------
\textbf{2013:} & Attended the TCD Zoology and Botany Postgraduate Symposium. Were I\\
& presented "Metabolic rate and body size linked with perception of temporal information".\\
%-------------------------------
\textbf{2012:} & Attended the IsoEcol: International Conference on Applications of Stable Isotope\\
&Techniques to Ecological Studies, Brest, France. I presented my talk entitiled\\
&"Accounting for the process of foraging in source-level variation in isotopic\\
& mixing models".\\
%-------------------------------


\end{tabular}

\pagebreak 
%-------------------------------------
% Workshops
%--------------------------------------
\section{Workshops}
\raggedright\textbf{Workshops I have taught}\\
\begin{tabular}{ll}
%---------------------------------------
\textbf{2016:} & Ran two day workshop on "Using Bayesian approaches in comparative analysis"\\
&University College Cork, Cork.\\
%---------------------------------------
\textbf{2016:} & Taught course on "Phylogentic comparative methods using MCMCglmm".\\
& as part of the \href{http://www.demogr.mpg.de/En/education_career/international_advanced_studies_in_demography_3279/past_courses_3280/comparative_approaches_in_ecology_and_evolution_4708/default.htm}{"Comparative Approaches in Ecology and Evolution"} workshop.\\
& Max Planck Institute for Demographic Research (MPIDR), Rostock, Germany.\\
%---------------------------------------
\textbf{2015:} & Teaching Assistant on "Using the COMPADRE Plant Matrix Database\\
& for comparative plant demography" workshop. BES annual meeting, Edinburgh.\\
\end{tabular}


\raggedright\textbf{Workshops I have attended}\\
\begin{tabular}{ll}
%---------------------------------------
\textbf{8-10/1/2018:} & Royal Society Disparity workshop: Collaborative meeting developing ideas regarding\\ 
&morphological disparity. Chicheley Hall, UK.\\
%---------------------------------------
\textbf{30/3/2017:} & Invited participant of the Royal Irish Academy Masterclass series with\\
& with Prof. Christian Stenseth.\\
%---------------------------------------
\textbf{05/10/2016:} & Individual stochastiticy: An introduction to demographic models and analysis,\\ 
&Hal Caswell, University of Virginia.\\
%---------------------------------------
\textbf{2015:} & Methods in Ecology and Evolution Workshop on Open Science, Darwin House London.\\
%---------------------------------------
\textbf{2014:} & Tansley Workshop: Collaborative meeting to develop metrics to measure ecosystem\\
&multistabilty, Silwood Park, Imperial College London.\\
%---------------------------------------
\textbf{2014:} & Software Carpentry Workshop covering Unix, Git repositories and creating\\
&R packages, University of Nottingham.\\
%---------------------------------------
\textbf{2013:} & Spatial Analysis in R Workshop, Barry Rowlingson, University of Sheffield.\\
%------------------
\textbf{2013:} & Introduction to Morphometrics Workshop, François Gould, Trinity College Dublin.\\
%---------------------
\textbf{2013:} & IUCN Red List of Ecosystems Workshop, Edmund Barrow, Trinity College Dublin.\\
%--------------------
\textbf{2012:} & Introduction to Bayesian analysis using WinBugs, David Lund, University of Cambridge.\\
%--------------------
\textbf{2012:} & Innovation Academy Creative thinking workshop, Trinity College Dublin.\\
%--------------------
\textbf{2012:} & Innovation Academy Film production workshop, Trinity College Dublin.\\
%--------------------
\textbf{2012:} & Introduction to the website management software DreamWeaver, Trinity College Dublin.\\
%--------------------
\textbf{2011:} & Introduction to Stable Isotope Mixing models, Andrew Jackson, Trinity College Dublin.\\
%--------------------
\textbf{2009:} & Mayfly Identification workshop, Mary Kelly Quinn, National Biodiversity Data Centre.\\
%--------------------
&\\
\end{tabular}






\bigskip

%-------------------------------------
% Publications
%--------------------------------------
\section{Publications}
\begin{flushleft}


\setlength{\parindent}{0mm}\textbf{Healy, K.,} Guillerme, T., Kelly, S.B.A., Inger, R., Bearhop, S., Jackson, A.L. 2018. SIDER: an R package for predicting trophic discrimination factors of consumers based on their ecology and phylogenetic relatedness. Ecography. {doi:10.1111/ecog.03371 Link to paper}.
\smallskip
\par{\fontsize{10.5}{10} Lead author: I developed a method to estimate trophic discrimination factors for the use in stable isotope dietary reconstruction models. This is available as an R package and is hosted on GitHub https://github.com/healyke/SIDER . This paper has 4 google scholar citations.}

\bigskip

\setlength{\parindent}{0mm} Adam Kane, \textbf{Healy, K.,}, Guillerme, T., Ruxton, G., and Jackson A.L. 2016. A recipe for scavenging and natural history. \textit{Ecography}. doi: [10.1111/ecog.02817]
\smallskip
\par{\fontsize{10.5}{10} I contributed to developing a framework to predict the importance of scavenging in extinct species using comparative physiology, ecological modeling and metabolic theory. This publication has 3 google scholar citation.}

\bigskip

\setlength{\parindent}{0mm}Kane, A., \textbf{*Healy, K.,} Ruxton, G.D., and Jackson, A.L. 2016. Body size drives importance of scavenging in theropods. \textit{The American Naturalist}. \textbf{6} (187), 706-716. \href{https://www.researchgate.net/profile/Kevin_Healy/publication/301279301_Body_Size_as_a_Driver_of_Scavenging_in_Theropod_Dinosaurs/links/570f8b2a08ae38897ba19c35.pdf.}{DOI: 10.1086/686094 Link to paper}
\smallskip
\par{\fontsize{10.5}{10} As corresponding and *Co-first author, We show that theropod dinosaurs of intermediate body size were more efficient scavengers than individuals at the extreme ranges of body sizes. We achieved this through using a novel individual based modeling approach that was parameterised using biomechanical models of theropod locomotion. I co-conceived the idea and carried out the data collection, analysis and writing of the paper. This publication has 3 google scholar citation.}

\bigskip

\setlength{\parindent}{0mm}Donohue, I., Hillebrand, H., Montoya, J.M., Petchey, O.L., Pimm, S.L., Fowler, M.S., \textbf{Healy, K.,} Jackson, A.L., Lurgi, M., McClean, D., O'Connor, N.E., O'Gorman, E.J., Yang, Q. 2016
 A., \textbf{Healy, K.,} Ruxton, G.D., and Jackson, A.L. 2016. Navigating the complexity of ecological stability. \textit{Ecology Letters}. 19 (9), 1172-1185.
\smallskip
\par{\fontsize{10.5}{10} Reviews and Syntheses article outlining the need to create a consistent language for ecological stability from the perspective of theoreticians, empiricists and policy makers. I contributed to this paper by developing methods to quantify overall ecological stability for management proposes that integrating across multiple dimensions to provide a single stability score. This publication has 31 google scholar citations.}

\bigskip

\textbf{Healy, K}., Guillerme, T., Finlay, S., Kane, A., Kelly, S.B.A., McClean, D., Kelly, D.J., Donohue, I., Jackson, A.L. and Cooper, N., 2014. Ecology and mode-of-life explain lifespan variation in birds and mammals. \textit{Proceedings of the Royal Society B}, \textbf{281}(1784), 20140298. \href{http://rspb.royalsocietypublishing.org/content/281/1784/20140298}{DOI:10.1098/rspb.2014.0298. Link to paper}.
\smallskip
\par{\fontsize{10.5}{10} As lead author I developed and carried out the main analysis along with data collection and writing of the manuscript. This paper showed that birds and mammals that either fly, forage underground or are arboreal live longer than expected for their size. This publication received significant media attention both nationally (e.g. the Moncrieff show, Irish Independent) and internationally (Discovery Channel) and has 59 google scholar citations.}

\bigskip

\textbf{Healy, K}., McNally, L, Ruxton, G., Cooper, N. and Jackson, A.L. 2013. Metabolic rate and\\
body size linked with perception of temporal information.  \textit{Animal Behaviour}. \textbf{86}, 685-696. \href{http://dx.doi.org/10.1016/j.anbehav.2013.06.018}{DOI:10.1016/j.anbehav.2013.06.018. Link to paper}.
\smallskip
\par{\fontsize{10.5}{10} As lead author I developed and carried out the main analysis along with data collection and writing of the manuscript. I showed that small animals with high metabolic rates process temporal information faster than large species with low metabolic rates. This was extensively covered in the media including coverage from 
\href{https://www.theguardian.com/science/2013/sep/16/time-passes-slowly-flies-study}{The Guardian}, 
\href{http://www.economist.com/news/science-and-technology/21586532-small-creatures-fast-metabolisms-see-world-action-replay-slo-mo}{The Economist},
 an appearance in BBC World News and has the highest ever \href{http://www.altmetric.com/details.php?key=517059da36b98ab7d4941284da32e5f7&citation_id=1705703&embedded=true}{alt-metric score} for this journal. This publication has 64 google scholar citations.} 

\bigskip

\setlength{\parindent}{0mm}Donohue, I., Petchey, O.L., Montoya, J.M., Jackson, A.L., McNally, L., Viana, M., \textbf{Healy, K}., Lurgi, M., O’Connor, N.E. and Emmerson, M.C. 2013. On the dimensionality of ecological stability. \textit{Ecology Letters}. \textbf{16}, 421-429. \href{http://onlinelibrary.wiley.com/doi/10.1111/ele.12086/abstract} {DOI:10.1111/ele.12086. Link to paper}. 
\smallskip
\par{\fontsize{10.5}{10} I co-developed the conceptual framework and statistical analysis used to produce the multidimensional ellipsoids and contributed to writing the manuscript.This publication has 92 google scholar citations.}
\bigskip


\textbf{Comment response}\\
\setlength{\parindent}{0mm}\textbf{Healy, K}. 2015.  Eusociality but not fossoriality drives longevity in small mammals. \textit{Proceedings of the Royal Society B}, \textbf{282}, 20142917. \href{http://rspb.royalsocietypublishing.org/content/282/1806/20142917} {DOI:10.1098/rspb.2014.2917. Link to paper}. 
\smallskip
\par{\fontsize{10.5}{10} Single author. I carried out additional analysis in response to a comment on my 2014 paper where I show eusociality but not fossoriality is a driver of longevity in mammals. This publication has 4 google scholar citations.}


\textbf{Statistical packages}\\
\setlength{\parindent}{0mm} Guillerme, T., \textbf{Healy, K.,} 2014. mulTree: a package for running MCMCglmm analysis on multiple trees. ZENODO. DOI: 10.5281/zenodo.12902 {https://github.com/TGuillerme/mulTree Link to package}.


\bigskip

\textbf{In review}\\

\setlength{\parindent}{0mm}\textbf{Healy, K.,} Carbone, C., and Jackson, A.L. Venom evolution in snakes is driven by body size, habitat dimensionality and a diet of eggs. In review in PNAS.
\smallskip
\par{\fontsize{10.5}{10} As lead author I collected the most complete dataset of physiological and ecological data relating to venomous snake species to date and show that the evolution of venom toxicity and volume are driven by snake body size, diet and the dimensionality of the habitat. I conceived the concept, collected the data, performed the analysis and wrote the manuscript.}
\smallskip

\setlength{\parindent}{0mm}\textbf{Healy, K.,} Salguero-Gómez, R., and Buckley, Y.M. Mapping animal life-history strategies using the COMADRE database. Target journal Nature Ecology and Evolution.
\smallskip
\par{\fontsize{10.5}{10} I used the COMADRE population matrix model database to show that animal life history strategies follow predictions based on metabolic theory and in general follows a separate axis of demography separate from the classical fast slow continuum. I performed the analysis and co-wrote the manuscript.}

\bigskip

\end{flushleft}

%-------------------------------------
% References
%--------------------------------------
%\section{References}


%\begin{tabular}{lcr}
% Referee 1
%\begin{minipage}[t]{2.2in}
%\textbf{Dr\ Andrew L. Jackson}\\
%Zoology Department\\
%School of Natural Sciences\\
%Trinity College Dublin\\
%Dublin 2\\
%Ireland\\
%Tel: + 353 1 896 2728\\
%Email:\href{mailto:a.jackson@tcd.ie}{a.jackson@tcd.ie}
%\end{minipage}

%&

%\begin{minipage}[t]{2.2in}
%\textbf{Dr\ Natalie Cooper}\\
%Life Sciences Department\\
%Natural History Museum\\
%Cromwell Road\\
%London\\
%SW7 5BD UK\\
%Tel: 0207942 5083\\
%Email:\href{mailto:natalie.cooper@nhm.ac.uk}{natalie.cooper@nhm.ac.uk}
%\end{minipage}


%\end{tabular}


%\bigskip

\end{document}