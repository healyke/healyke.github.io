% 10/06/2017 
%LaTex version of my CV

\documentclass[10pt,a4paper]{article}
\usepackage[left=0.7in, top=0.6in, right=0.7in, bottom=0.6in]{geometry}
\usepackage{enumerate} % put in numbers or bullet points
\usepackage{setspace}	% line spacing					
\usepackage{eurosym}
%\usepackage{fullpage}
\usepackage{fancyhdr}
\pagestyle{fancy} % page numbers and headers and footers

\usepackage{lipsum}

\newcommand\textbox[1]{%
  \parbox{.333\textwidth}{#1}%
}


\usepackage{amsmath}

\newcommand{\textoperatorname}[1]{%
  \operatorname{\textnormal{#1}}%
}




\fancyhead[RO,RE]{Dr Kevin Healy CV}
\fancyfoot[C]{\thepage~of 4}

\renewcommand{\headrulewidth}{0.1pt}
\renewcommand{\footrulewidth}{0pt}

\renewcommand*{\familydefault}{\sfdefault} %% I changed the font
\usepackage{helvet}

%\usepackage[osf]{mathpazo} % palatino font package
%\usepackage{fontspec} %this must be in XeLaTeX
%\setmainfont{Verdana}%
%\setmainfont{Calibri}%
%\setsansfont{Tahoma}%
%\setmainfont{Helvetica}%
%\setmainfont{Garamond}%
\usepackage{hyperref} % allows the inclusion of hyperlinks




\usepackage{titlesec} % Used to customize the \section command
\titleformat{\section}{\Large\raggedright}{}{0em}{} [\titlerule] % Text formatting of sections
\titlespacing{\section}{0pt}{3pt}{10pt} % Spacing around sections (left, above, below)

\begin{document}

\par{\centering{\Huge \textbf{Dr Kevin {Healy}}}\smallskip\par}

\large\centering{{Macroecologist at the University of St Andrews,\\
Marie Sk\l{}odowska-Curie Individual Fellow,\\
Sir Harold Mitchell Building,\\
St Andrews, UK}}\\
\bigskip

\noindent\textbox{h214@st-andrews.ac.uk}\textbox{\hfil \href{http://scholar.google.com/citations?user=5Kb9u8EAAAAJ}{Google Scholar} \hfil}\textbox{\hfill \href{http://healyke.github.io}{healyke.github.io}}


\bigskip


\section{\textbf{Research Interests}}
\center	
My research focuses on macroecological patterns in the life history and trophic ecology of animals. I use a range of comparative and theoretical approaches, including developing packages for the R statistical language.%DM added comma after approaches
\bigskip



\section{\textbf{Education and Academic Career}}

\raggedright	
\textbf{2017 - current:} Marie Sk\l{}odowska-Curie Individual Fellow, The University of St Andrews.
 \smallskip
\par{\fontsize{10.5}{10} My SCAVENGER project aims to develop agent based models to test the role of body mass and other factors in driving the evolution and ecology of scavenging behaviours and traits. % DMscavenging foraging sounds weird- I am sure it is right it just doesn't sound it, foreging through scavenging maybe? Just thinking aloud though
\bigskip}

\raggedright	
\textbf{2015 - 2017:} Research Fellow in Zoology, Trinity College Dublin.
 \smallskip
\par{\fontsize{10.5}{10} I tested questions regarding how body size, ancestral relationships, and ecology determine species' life history strategies in Professor Yvonne Buckley's research group. Funded by Science Foundation Ireland.\bigskip}%DM added apostrphe and changed spelling of research! 

\raggedright	
\textbf{2011 - 2015:} Ph.D. in Zoology, Trinity College Dublin.\\
Thesis title: \textit{Predator-prey allometry across body size and interaction dimensionality.}\\ \par{\fontsize{10.5}{10} Supervised by Dr. Andrew Jackson and Dr. Andrew Parnell as part of the  Earth and Natural Sciences Doctoral Studies Programme.
\bigskip}


%------------------------------------------

\textbf{2007-2011:} B.A. Mod in Zoology, First class honours (77\%), Trinity College Dublin.\\
Thesis: \textit{Fractal structure of intestinal parasite communities in field mice.} ( 82\%).\\  
Involved mammal trapping in the field, dissection work, and parasite identification.

\bigskip

%------------------------------------------
\textbf{2010:} Ureka research position in SoMER program, National College of Ireland Maynooth.\\
\bigskip

%------------------------------------------
\textbf{2006-2007:} General archaeological operator for Valerie J. Keeley Ltd.\\
\bigskip




%	Awards and Grants
%----------------------------------------------------------------------------------------

\section{\textbf{Awards and Grants}}


\begin{tabular}{ll}
\textbf{2017:} & Marie Sk\l{}odowska-Curie Individual Fellowship (MSCA-IF). Funded by Horizon 2020\\
& (\euro 183,454.80).\\
\textbf{2017:} & Government of Ireland Postdoctoral Fellowship. Funded by the Irish Research\\
& Council (\euro 91,330) *I declined this award to accept the MSCA-IF.\\
\textbf{2014:} & Gordon Research Seminar mentoring program position. Funded by Gordon\\
& Research Conferences and the National Science Foundation. (\euro 1,300)\\
\textbf{2016:} & Winner of the Postdoc category of the Trinity College Dublin, School of Natural\\
& Sciences lightning talks.\\
\textbf{2014:} & Awarded runner up in both the TCD School of Natural Sciences postgraduate\\
& lightning talks and at the joint Zoology and Botany postgraduate symposium.\\
\textbf{2014:} & Awarded runner up in the "I'm a scientist get me out of here" outreach competition\\
\textbf{2014:} & Awarded prize of top contributor in the research discussion group "NERD club".\\
\textbf{2011:} & Awarded Gold medal for "��exceptional merit at degree examinations"�� in final\\
& year of B.A Mod. Zoology.
\end{tabular}

\bigskip


%-------------------------------------
% Publications
%--------------------------------------
\section{\textbf{Publications}}
\begin{flushleft}


\setlength{\parindent}{0mm}\textbf{Healy, K.,} Guillerme, T., Kelly, S.B.A., Inger, R., Bearhop, S., Jackson, A.L. 2017. SIDER: an R package for predicting trophic discrimination factors of consumers based on their ecology and phylogenetic relatedness. \textit{\textbf{Ecography}}. \href{https://onlinelibrary.wiley.com/doi/abs/10.1111/ecog.03371}{doi:10.1111/ecog.03371 Link to paper}. 7 Google scholar citations.
\bigskip

\setlength{\parindent}{0mm} Adam Kane, \textbf{Healy, K.,}, Guillerme, T., Ruxton, G., and Jackson A.L. 2017. A recipe for scavenging and natural history. \textit{\textbf{Ecography}}. \href{https://onlinelibrary.wiley.com/doi/pdf/10.1111/ecog.02817}{doi:10.1111/ecog.02817 Link to paper}. 4 Google scholar citations.
\bigskip


\setlength{\parindent}{0mm}Kane, A., \textbf{*Healy, K.,} Ruxton, G.D., and Jackson, A.L. 2016. Body size drives importance of scavenging in theropods. \textit{\textbf{The American Naturalist}}. \textbf{6} (187), 706-716. \href{https://www.researchgate.net/profile/Kevin_Healy/publication/301279301_Body_Size_as_a_Driver_of_Scavenging_in_Theropod_Dinosaurs/links/570f8b2a08ae38897ba19c35.pdf.}{DOI: 10.1086/686094 Link to paper}. *Joint-First author. I co-conceived the idea and carried out the data collection, analysis and co-wrote of the paper. 4 Google scholar citations.

\bigskip

\setlength{\parindent}{0mm}Donohue, I., Hillebrand, H., Montoya, J.M., Petchey, O.L., Pimm, S.L., Fowler, M.S., \textbf{Healy, K.,} Jackson, A.L., Lurgi, M., McClean, D., O'Connor, N.E., O'Gorman, E.J., Yang, Q. 2016
 A., \textbf{Healy, K.,} Ruxton, G.D., and Jackson, A.L. 2016. Navigating the complexity of ecological stability. \textit{\textbf{Ecology Letters}}. 19 (9), 1172-1185. \href{https://onlinelibrary.wiley.com/doi/abs/10.1111/ele.12648}{doi:10.1111/ele.12648 Link to paper}.
\par{\fontsize{10.5}{10} 42 Google scholar citations.}

\bigskip

\textbf{Healy, K}., Guillerme, T., Finlay, S., Kane, A., Kelly, S.B.A., McClean, D., Kelly, D.J., Donohue, I., Jackson, A.L. and Cooper, N., 2014. Ecology and mode-of-life explain lifespan variation in birds and mammals. \textit{\textbf{Proceedings of the Royal Society B}}, \textbf{281}(1784), 20140298. \href{http://rspb.royalsocietypublishing.org/content/281/1784/20140298}{DOI:10.1098/rspb.2014.0298. Link to paper}. This publication received significant media attention both nationally (e.g. the Moncrieff show, Irish Independent) and internationally (Discovery Channel). 71 Google scholar citations.

\bigskip

\textbf{Healy, K}., McNally, L, Ruxton, G., Cooper, N. and Jackson, A.L. 2013. Metabolic rate and\\
body size linked with perception of temporal information.  \textit{\textbf{Animal Behaviour}}. \textbf{86}, 685-696. \href{http://dx.doi.org/10.1016/j.anbehav.2013.06.018}{DOI:10.1016/j.anbehav.2013.06.018. Link to paper}. This was extensively covered in the media including coverage from 
\href{https://www.theguardian.com/science/2013/sep/16/time-passes-slowly-flies-study}{The Guardian}, 
\href{http://www.economist.com/news/science-and-technology/21586532-small-creatures-fast-metabolisms-see-world-action-replay-slo-mo}{The Economist},
 an appearance in BBC World News. 64 Google scholar citations. 

\bigskip

\setlength{\parindent}{0mm}Donohue, I., Petchey, O.L., Montoya, J.M., Jackson, A.L., McNally, L., Viana, M., \textbf{Healy, K}., Lurgi, M., O’Connor, N.E. and Emmerson, M.C. 2013. On the dimensionality of ecological stability. \textit{\textbf{Ecology Letters}}. \textbf{16}, 421-429. \href{http://onlinelibrary.wiley.com/doi/10.1111/ele.12086/abstract} {DOI:10.1111/ele.12086. Link to paper}. 110 Google scholar citations.
\bigskip


\textbf{Comment response}\\
\setlength{\parindent}{0mm}\textbf{Healy, K}. 2015.  Eusociality but not fossoriality drives longevity in small mammals. \textit{\textit{Proceedings of the Royal Society B}}, \textbf{282}, 20142917. \href{http://rspb.royalsocietypublishing.org/content/282/1806/20142917} {DOI:10.1098/rspb.2014.2917. Link to paper}.  I carried out additional analysis in response to a comment on my 2014 paper where I show eusociality but not fossoriality is a driver of longevity in mammals. 5 Google scholar citations.

\bigskip

\textbf{Statistical packages}\\
\setlength{\parindent}{0mm} Guillerme, T., \textbf{Healy, K.,} 2014. mulTree: a package for running MCMCglmm analysis on multiple trees. ZENODO. DOI: 10.5281/zenodo.12902 {https://github.com/TGuillerme/mulTree Link to package}. This package is hosted on GitHub \href{https://github.com/TGuillerme/mulTree} and has been cited 8 times.

\bigskip

\clearpage

\textbf{Forthcoming Papers}\\

\setlength{\parindent}{0mm}\textbf{Healy, K.,} Ezard, T.H.G., and Jones, O.R., Salguero-G\'{o}mez, R., and Buckley, Y.M. Beyond the fast-slow continuum in animal life-histories. In review in \textit{\textbf{Nature Ecology and Evolution}}.

\bigskip

\setlength{\parindent}{0mm}\textbf{Healy, K.,} Carbone, C., and Jackson, A.L. Venom evolution in snakes is driven by body size, habitat dimensionality and a diet of eggs. Under consideration at \textit{\textbf{Ecology Letters}}.

\bigskip

\section{\textbf{Outreach, and academic service}}
\raggedright\textbf{Outreach}\\
\begin{tabular}{ll}
%---------------------------------------
\textbullet& I gave a TEDxUCD talk on lifespan evolution in animals in 2015 (\href{https://www.youtube.com/watch?v=-CHtfWEKifY}{link to talk}).\\
\textbullet& I featured in the documentary "Superpowers of naked mole rats" which was broadcasted\\ 
& on French-German channel Arte early in 2018.\\
\textbullet& I co-organised three \href{http://www.ecoevoblog.com/2014/09/22/night-life-friday-26th-sept/}{Discover Research Night} events (2013-15) in the Zoology Museum,\\
& Trinity College Dublin, which received a combined attendance of over 600.\\ 
\textbullet& I have been involved in numerous outreach activities including blog contributions to\\ 
& the Journal of Animal Ecology \href{https://journalofanimalecology.wordpress.com/2017/09/23/high-society-the-social-network-of-vultures/}{(link)}, presenting at a PubPhD event, and partaking in\\
& the 2014 "I'm a scientist get me out of here" event were I was a finalist.\\
\end{tabular}

\raggedright\textbf{Academic service}\\
\begin{tabular}{ll}
%---------------------------------------
\textbullet& I am a committee member for the BES Macroecology Special Interest Group and currently\\
& on the organising team for the Special Interest Groups annual meeting in St Andrews 2018.\\
\textbullet& I am co-organising the symposium \href{http://evolutionmontpellier2018.org/symposia}{"Exploring life history evolution across multiple scales"}\\ 
& at the II Joint Congress on Evolutionary Biology in Montpellier, 2018.\\
\textbullet& Committee member for the Irish Ecological Association ,2016-2018.\\
\textbullet&Postgraduate representative for the Zoology Department, Trinity College Dublin, 2014-15.\\
\textbullet&I regularly act as a reviewer for international journals including Ecology Letters,\\
&Current Biology, Proceedings of the Royal Society B, Journal of Biogeography, etc\\
&and for the The German Israeli Foundation for Scientific Research and Development.\\
\end{tabular}

\bigskip

\section{\textbf{Teaching}}
\raggedright\textbf{Undergraduate}\\
\textit{University of St Andrews}
\begin{tabular}{ll}
\textbullet& Lecture: Introduction to the metabolic theory of ecology for the first year Biology Comparative\\
& Physiology module.\\%DM altered structure
\textbullet& Contributed to the development of a third year Biology module on Animal-Plant Interactions.\\
\end{tabular}%DMadded "a"

\smallskip
\textit{Trinity College Dublin}
\begin{tabular}{ll}
\textbullet& I developed and taught a Macroevolution series for the Zoology final year Evolution module.\\
\textbullet& I delivered a lecture on "Altruism" as part of the 2nd year Introduction to Evolution module.\\%DM added capital
\textbullet& I led Research Comprehension classes for final year Zoology students.\\%changed 4th to final
\textbullet& Field assistant for the 3rd year Zoology Terrestrial Ecology field course.\\ %DM  capitalised for consistency
\textbullet& I ran a walk-in thesis statistics help workshops for final years students.\\ 

\end{tabular}

\smallskip
\raggedright\textbf{Post Graduate and workshops}\\
\begin{tabular}{ll}
\textbullet& I designed and taught a Statistics for Biology module for the Biodiversity and Conservation\\ 
& masters programme in Trinity College Dublin.\\
\textbullet&  I designed and taught a two day comparative analysis workshop in University College Cork\\
&  and a module as part of the \href{http://www.demogr.mpg.de/En/education_career/international_advanced_studies_in_demography_3279/past_courses_3280/comparative_approaches_in_ecology_and_evolution_4708/default.htm}{"Comparative Approaches in Ecology and Evolution"} for the\\
& Max Plank Institute for Demographic Research, Rostock, Germany.\\
\textbullet& Teaching Assistant for the "Using the COMPADRE Plant Matrix Database\\
& for comparative plant demography" workshop. BES annual meeting, Edinburgh.\\
\end{tabular}
\raggedright\textbf{Student supervision}\\
\begin{tabular}{ll}
\textbullet&  I was a co-supervisor for three final year students in Trinity College Dublin and one in the\\
& University of St Andrews \\
\textbullet&  I am currently acting as a PhD co-supervisor to a student in Trinity College Dublin.\\
\end{tabular}


\bigskip


%-------------------------------------
% Conferences and workshops
%--------------------------------------


\section{\textbf{Conferences and public speaking}}

\textit{Invited and Keynote talks}
\begin{tabular}{ll}
%-------------------------------
\textbf{2016:} & Invited seminar speaker to the Animal and Plant Sciences Department in\\
& The University of Sheffield. "Mapping animal life-history strategies"\\
%-------------------------------
\textbf{2015:} & TEDxUCD speaker. \href{https://www.youtube.com/watch?v=-CHtfWEKifY}{"Listening to evolutionary oddities"}.\\
%-------------------------------
\textbf{2015:} & Invited speaker to the Dublin Science Gallery Dark Secrets event.\\ 
& "BIOLUMINESCE: How living organisms produce and emit light"\\ 
%-------------------------------
\textbf{2014:} & Keynote student speaker at the BES Macroecology meeting, Nottingham.\\ 
&"Ecology and mode-of-life explain lifespan variation in birds and mammals".\\
%-------------------------------
\textbf{2014:} & Invited speaker to the Irish Longitudinal Study on Aging group (TILDA).\\ 
& "Ecology and mode-of-life explain lifespan variation in birds and mammals".\\
%-------------------------------
\textbf{2014:} & Invited speaker to the Dublin Science Gallery DEAD BEATS event.\\ 
& "Why so venomous?".\\
%-------------------------------
\end{tabular}

\bigskip
I have also presented my work at numerous international conferences  including ESA in Portland 2017, annual BES meetings in 2016 and 2015, the Gordon Research Seminar "Unifying Ecology Across Scales" in 2014, the ESEB XIV Congress 2013, EvoDemos in Virgina 2016, IsoEcol in 2012 and the BES Macroecology Sig in 2013, 2014 and 2017.
%-------------------------------
\bigskip


\section{Professional Training}
\begin{tabular}{ll}
\textbf{2016:} & Passport to Research Futures: Structured development programme aimed at\\
& professional development for early career researchers. University of St Andrews.\\
%------------------
\textbf{2016:} & Individual stochastiticy: An introduction to demographic models and analysis,\\ 
&Hal Caswell, University of Virginia.\\
%------------------
\textbf{2015:} & Methods in Ecology and Evolution Open Science workshop, Darwin House, London.\\
%------------------
\textbf{2014:} & Software Carpentry Workshop covering Unix, Git repositories and creating\\
&R packages, University of Nottingham.\\
%------------------
\textbf{2013:} & Spatial Analysis in R Workshop, Barry Rowlingson, University of Sheffield.\\
%------------------
\textbf{2013:} & Introduction to Morphometrics Workshop, Fran\c{c}ois Gould, Trinity College Dublin.\\
%---------------------
\textbf{2013:} & IUCN Red List of Ecosystems Workshop, Edmund Barrow, Trinity College Dublin.\\
%--------------------
\textbf{2012:} & Introduction to Bayesian analysis using WinBugs, David Lund, University of\\
&Cambridge.\\
%--------------------
\textbf{2012:} & Innovation Academy Creative thinking workshop, Trinity College Dublin.\\
%--------------------
\textbf{2012:} & Innovation Academy Film production workshop, Trinity College Dublin.\\
%--------------------
\textbf{2009:} & Mayfly Identification workshop, Mary Kelly Quinn, National Biodiversity Data Centre.\\
\end{tabular}

\bigskip

%-------------------------------------
% Workshops
%--------------------------------------
\section{Working research groups}

\begin{tabular}{ll}
\textbf{2018:} & sBioRange: Working group aimed at assembling the first global database of the\\
& strength of interactions between species. Proposal for sDiv working group currently\\
& under review.\\
%---------------------------------------
\textbf{2018:} & Royal Society Disparity workshop: Collaborative meeting developing ideas\\ 
& relating to morphological disparity. Chicheley Hall, UK.\\
%---------------------------------------

\textbf{2018:} & Timespines working group. Collaborative research group using paleontological\\
& data to elucidate the drivers of the evolution of defensive structures in animals.\\
%---------------------------------------


\textbf{2018:} & Biogeogrpahy working group. Collaborative group using data compiled from the\\
& literature to test the role if isolation in the evolutionary island syndrome\\ 
%---------------------------------------
\textbf{2014-16:} & Tansley Workshop: Collaborative meetings to develop metrics to measure \\
& ecosystem multistabilty, Silwood Park, Imperial College London.\\
%---------------------------------------

%---------------------------------------

%--------------------
&\\
\end{tabular}




\end{flushleft}

%-------------------------------------
% References
%--------------------------------------
%\section{References}


%\begin{tabular}{lcr}
% Referee 1
%\begin{minipage}[t]{2.2in}
%\textbf{Dr\ Andrew L. Jackson}\\
%Zoology Department\\
%School of Natural Sciences\\
%Trinity College Dublin\\
%Dublin 2\\
%Ireland\\
%Tel: + 353 1 896 2728\\
%Email:\href{mailto:a.jackson@tcd.ie}{a.jackson@tcd.ie}
%\end{minipage}

%&

%\begin{minipage}[t]{2.2in}
%\textbf{Dr\ Natalie Cooper}\\
%Life Sciences Department\\
%Natural History Museum\\
%Cromwell Road\\
%London\\
%SW7 5BD UK\\
%Tel: 0207942 5083\\
%Email:\href{mailto:natalie.cooper@nhm.ac.uk}{natalie.cooper@nhm.ac.uk}
%\end{minipage}


%\end{tabular}


%\bigskip

\end{document}